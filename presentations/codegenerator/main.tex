\documentclass[department=icis, slidenumbers=slide, official=true]{beamerruhuisstijl}
\title{Code Generation in Haskell}
\subtitle{Aaron van Geffen and Thom Wiggers}
\date{\today}
\author{Aaron van Geffen \and Thom Wiggers}

\usepackage{hyperref}
\usepackage{graphicx}
\usepackage{microtype}
\usepackage{cleveref}
\usepackage{color}
\usepackage{minted}
\usepackage{multicol}
\usepackage{ulem}

\usemintedstyle{friendly}

\begin{document}

\begin{frame}
    \titlepage{}
\end{frame}

% * Problemen met Type Inference benoemen
% 	* Eerst afgemaakt voor aan de code generation begonnen werd

% * Problemen met overloading: hoe gaan we print doen?

% * Conclusie [ergens]: oeps, we hadden een geannoteerde AST moeten maken


% * Typesysteem van Haskell benutten voor vertaalslag: IR en SSM

% * Vertalen naar straight-line program in IR


% * Register-allocatie: praktisch niet nodig, bijna alle OPs in SSM werken op stack

% * ARM64 / AArch64 voor de volgende keer -> Cortex A53


\begin{frame}{Outline}
    \tableofcontents
\end{frame}


\section{State of Affairs}
\begin{frame}{State of Affairs} % State of the SPLunion
    \pause
    \begin{itemize}[<+->]
        \item Off to a rocky start \ldots
        \item We had to finish our type checker, still.
            \vspace{1em}

        \item Basic type inference had already been implemented \ldots
        \item ... but we were still struggling with overloading: how to do `print`?
            \vspace{1em}

        \item Loads of `with the knowledge we have today' moments \ldots
        \item We should probably have built an annotated AST.
            \vspace{1em}

        \item \ldots
    \end{itemize}
\end{frame}


\section{From AST to Intermediate Representation}
\begin{frame}{Intermediate Representation}
    \fontsize{8pt}{12}\selectfont
    \pause
    \inputminted{Haskell}{examples/ir_instructions.hs}
\end{frame}

\section{From IR to Simple Stack Machine}
\begin{frame}{Simple Stack Machine}
    \fontsize{8pt}{9}\selectfont

    % \pause
    % \begin{multicols}{3}
    %     \inputminted{Haskell}{examples/newtypes.hs}
    % \end{multicols}

    \pause
    \begin{multicols}{3}
        \inputminted{Haskell}{examples/ssm.hs}
    \end{multicols}

\end{frame}

\begin{frame}{Writer Monad}
    \pause
    \inputminted{Haskell}{examples/ssm_monad.hs}
\end{frame}

\begin{frame}{Example: binary operators}
    \fontsize{8pt}{9}\selectfont
    \pause
    \inputminted{Haskell}{examples/binop_to_ssm.hs}
\end{frame}

\begin{frame}{Converting SPL into Straight-line Program}
    \vspace{30pt}
    Convert this: \\
    \texttt{r = f(g(x), h(y), k(z));}

    \vspace{30pt}
    \only<2>{
        Into this: \\
        \texttt{a = g(x); \\
                b = h(y); \\
                c = k(z); \\
                r = f(a, b, c); }
    }
\end{frame}

\section{Next project}
\begin{frame}{Next project}
    \begin{itemize}[<+->]
        \setlength\itemsep{1em}

        \item
            \only<1>{
                Initial idea: let's build this on LLVM!
            }
            \only<2->{
                \sout{Initial idea: let's build this on LLVM!}
            }
        \item
            Actually, let's use these Cortex A53 we have lying around!
        \item
            AArch64 (ARM64) architecture.
        \item
            Challenges\ldots
            \begin{itemize}[<+->]
                \item Real hardware!
                \item Register allocation
                \item Debugging?
            \end{itemize}
    \end{itemize}
\end{frame}

\begin{frame}{Closing words}
    \begin{itemize}[<+->]
        \setlength\itemsep{1em}

        \item No regrets for choosing Haskell as the programming language!

        \item Still (nice) challenges ahead for the actual code generation.

        \item Optimistic about generating code for ARM64.

        \item Stay tuned for a nice demo next time. \texttt{:-)}
    \end{itemize}
\end{frame}

\end{document}

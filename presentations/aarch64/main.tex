\documentclass[department=icis, slidenumbers=slide, official=true]{beamerruhuisstijl}
\title{SPL on AArch64}
\subtitle{Aaron van Geffen and Thom Wiggers}
\date{\today}
\author{Aaron van Geffen \and Thom Wiggers}

\usepackage{hyperref}
\usepackage{graphicx}
\usepackage{microtype}
\usepackage{cleveref}
\usepackage{color}
\usepackage{minted}

\usemintedstyle{friendly}

\begin{document}

\begin{frame}
    \titlepage{}
\end{frame}

\begin{frame}{Why run SPL on ARM?}
    \begin{itemize}
        \item Seemed like a fun challenge
        \item No more emulators in Java
        \item Real register-based architecture and calling convention
        \item Thom has experience with ARM
    \end{itemize}
\end{frame}

\begin{frame}{Real Hardware}
    \begin{columns}
        \column{0.55\textwidth}
        \begin{figure}[h]
            \centering
            \includegraphics[width=0.8\textwidth]{imgs/odroid.jpg}
            \caption{ODROID-C2}\label{fig:odroid}
        \end{figure}

        \column{0.45\textwidth}
        \begin{itemize}
            \item Quad-Core CPU
            \item 64-bit Cortex-A53
            \item ARMv8a
            \item 1536 MHz
            \item 2GB RAM
            \item Low Power
            \item \$ 45
        \end{itemize}
    \end{columns}
\end{frame}

\begin{frame}{AArch64}
    \begin{itemize}
        \item 64-bit architecture mode of ARMv8a
        \item RISC architecture
        \item 31 general-purpose registers $r0, \ldots, r31$
            \begin{itemize}
                \item Does not include the program counter or stack pointer ($sp$)
                \item Notation: $x0$ is $r0$ in 64-bit mode, $w0$ is $r0$ in 32-bit mode
            \end{itemize}
        \item Instructions have discrete inputs and outputs. They can also get one immediate values or receive their input with some rotation.
            \begin{itemize}
                \item \texttt{add Rd, Rn, Rm}
                \item \texttt{add Rd, Rn, \#10}
                \item \texttt{add Rd, Rn, Rm, LSL 10}
            \end{itemize}
        \item Branches go through \emph{Link Register}
        \item Advanced SIMD capabilities: 32 128-bit vector registers (not used in SPL)
    \end{itemize}
\end{frame}

\begin{frame}{Going from SSM to AArch64}
    \begin{columns}
        \column{0.45\textwidth}
        \begin{block}{SSM}
            \begin{itemize}
                \item No registers used for operations
                \item Special operations for I/O
                \item Heap grows organically
                \item No separation of code and data segments
            \end{itemize}
        \end{block}
        \column{0.45\textwidth}
        \begin{block}{AArch64}
            \begin{itemize}
                \item Mandatory registers for function calling
                \item System calls needed for I/O
                \item Heap allocated manually
                \item Specific code and data segments
            \end{itemize}
        \end{block}
    \end{columns}
\end{frame}

\begin{frame}{Global Variables}
    \begin{itemize}
        \item Known at compile time
        \item In C they live on the heap\ldots
        \item \ldots but initial value needs to be known at compile time (static)
        \item In SPL, any expression is allowed
        \item Move initialisation code into special function at compile-time.

        \pause
        \vspace{1em}

        \item In SSM, we used a register to locate globals in memory
        \item For AArch64, we specify labeled \texttt{.data} sections in the generated assembly code 
        \item Globals are accessed by loading the address of that label
    \end{itemize}
\end{frame}

\begin{frame}{AArch64 Calling Convention}
    We adopted ARM's C calling conventions\footnote{\url{http://infocenter.arm.com/help/topic/com.arm.doc.ihi0055b/IHI0055B_aapcs64.pdf}} for function calls:
    \begin{itemize}
        \item Registers $r0, \ldots, r18$ are caller-saved
        \item Registers $r19, \ldots, r30$ are callee-saved
        \item Registers $r0, \ldots, r7$ are used for arguments
        \item Extra arguments are put on the stack
        \item The Link Register ($r30$) contains the return address
        \item Register $r0$ contains the function result after return
    \end{itemize}

    \vspace{3em}
    Allows easy use of \texttt{libc} library functions
\end{frame}

\begin{frame}{Register Allocation}
    \begin{block}{Typical Approach}
    \begin{enumerate}[<+->]
        \item Divide the IR into \emph{Basic Blocks}
        \item Create a \emph{Static Single Assignment} version of the IR
        \item Create dependency graph
        \item Apply a graph colouring algorithm
        \item When necessary, generate spill code and go back to step 2
        \item Finally, assign registers accordingly
    \end{enumerate}
    \end{block}
    \vspace{2em}
    \only<7->{Both step 2 and 4 are not trivial.
    We opted to do a more naive approach first.}

\end{frame}

\begin{frame}{Register Allocation}
    \begin{block}{Our Approach}
        \begin{enumerate}[<+->]
            \item Divide the IR into \emph{Basic Blocks}
            \item Go through the IR bottom up, for each instruction:
                \begin{enumerate}
                    \item Collect the output pseudo-register and input pseudo-registers
                    \item Remove the output register from the current list of live registers
                    \item Add the input registers to the list
                    \item Annotate the instruction with the current list
                    \item Move on to previous line
                \end{enumerate}
            \item Use the information about live pseudo-registers to assign them to hardware registers
        \end{enumerate}
    \end{block}
\end{frame}

\begin{frame}{Current Status}
    \begin{itemize}[<+->]
        \item All instructions are implemented
        \item Tuples and lists are allocated on a statically sized heap
        \item \texttt{print} and \texttt{printInt} call \texttt{libc}'s \texttt{putwchar} and \texttt{printf}
        \item We can run recursive functions, however\ldots
        \item Current register allocation has problems:
            \begin{itemize}
                \item Hard to incorporate calling convention
                \item Cannot generate register spills
                \item Far from optimal: generates unnecessary moves
            \end{itemize}
    \end{itemize}
\end{frame}

\begin{frame}
    \begin{figure}[h]
        \centering
        \includegraphics[width=0.8\textwidth]{imgs/lego.jpg}
        \caption{A Message From Our Sponsors}\label{fig:lego}
    \end{figure}
\end{frame}

\end{document}

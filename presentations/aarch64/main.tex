\documentclass[department=icis, slidenumbers=slide, official=true]{beamerruhuisstijl}
\title{SPL on AArch64}
\subtitle{Aaron van Geffen and Thom Wiggers}
\date{\today}
\author{Aaron van Geffen \and Thom Wiggers}

\usepackage{hyperref}
\usepackage{graphicx}
\usepackage{microtype}
\usepackage{cleveref}
\usepackage{color}
\usepackage{minted}

\usemintedstyle{friendly}

\begin{document}

\begin{frame}
    \titlepage{}
\end{frame}

\begin{frame}{Why run SPL on ARM?}
    \begin{itemize}
        \item Thom has lots of experience with ARM
        \item No more silly emulators in Java
        \item Seemed like fun
    \end{itemize}
\end{frame}

\begin{frame}{Real Hardware}
    \begin{columns}
        \column{0.5\textwidth}
        Plaatje

        \column{0.5\textwidth}
        \begin{itemize}
            \item Quad-Core CPU
            \item 64-bit Cortex-A53
            \item ARMv8a
            \item 1.536 MHz
            \item 2GB RAM
            \item Low Power
            \item \$ 45
        \end{itemize}
    \end{columns}
\end{frame}

\begin{frame}{AArch64}
    \begin{itemize}
        \item 64-bit architecture mode of ARMv8a
        \item RISC architecture
        \item 31 general-purpose registers $r0, \ldots, r31$
            \begin{itemize}
                \item Doesn't include the program counter or stack pointer ($sp$)
                \item Notation: $x0$ is $r0$ in 64-bit mode, $w0$ is $r0$ in 32-bit mode
            \end{itemize}
        \item Instructions have discrete inputs and outputs. They can also get one immediate values or receive their input with some rotation.
            \begin{itemize}
                \item \texttt{add Rd, Rn, Rm}
                \item \texttt{add Rd, Rn, \#10}
                \item \texttt{add Rd, Rn, Rm, LSL 10}
            \end{itemize}
        \item Branches go through \emph{Link Register}
        \item Advanced SIMD capabilities: 32 128-bit vector registers (not used in SPL)
    \end{itemize}
\end{frame}

\begin{frame}{AArch64 Calling Convention}
    We adopted ARM's C calling conventions\footnote{\url{http://infocenter.arm.com/help/topic/com.arm.doc.ihi0055b/IHI0055B_aapcs64.pdf}} for function calls:
    \begin{itemize}
        \item Registers $r0, \ldots, r7$ are used for arguments
        \item Registers $r0, \ldots, r18$ are caller-saved
        \item Registers $r19, \ldots, r30$ are callee-saved
        \item Extra arguments are put on the stack
        \item The Link Register ($x30$) contains the return address
        \item Register $r0$ contains the function result after return
    \end{itemize}

    \vspace{3em}
    Allows easy use of libc library functions
\end{frame}

\begin{frame}{Going from SSM to AArch64}
    \begin{columns}
        \column{0.5\textwidth}
        \begin{block}{SSM}
            \begin{itemize}
                \item No registers used for operations
                \item Special operations for I/O
                \item Heap grows organically
                \item No separation of code and data segments
            \end{itemize}
        \end{block}
        \column{0.5\textwidth}
        \begin{block}{AArch64}
            \begin{itemize}
                \item Mandatory registers for function calling
                \item System calls needed for I/O
                \item Heap allocated manually
                \item Specific code and data segments
            \end{itemize}
        \end{block}
    \end{columns}
\end{frame}

\end{document}

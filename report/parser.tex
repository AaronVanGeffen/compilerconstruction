\chapter{Parser}
\todo[inline]{Dibs Aaron}

\section{Changes to the grammar}

In order to successfully implement an $LL$ parser, we first needed to eleminate left-recursive elements from the original SPL grammar.
The elements in question were located in the \textsf{FArgs}, \textsf{Field}, and \textsf{Exp} rules.

For a complete overview of the implemented grammar, please see Appendix \ref{ch:grammar}.
This section will detail the the changes made to the original grammar.

\subsection{Removing left recursion}

The first two rules, \textsf{FArgs} and \textsf{Field}, were trivial to change.
Hence, the transformed rules read as follows:

\begin{framed}
	\begin{grammar}
	<FArgs> ::= <id> [ `,' <FArgs> ]

	<Field> ::= [ `.' (`hd' | `tl' | `fst' | `snd') <Field> ]
	\end{grammar}
\end{framed}

However, the \textsf{Exp} field required more work to transform its binary operations.

\subsection{Transforming binary operations}

The original grammar defined binary operations within the \textsf{Exp} rule as a combination of two expressions and an operator: \textsf{Exp} \textsf{Op2} \textsf{Exp}.
We dealt with the left recursion here by unrolling the binary expressions, taking their operator precedence into account:

\begin{framed}
	\begin{grammar}
		<Exp> ::= <Exp2> `:' <Exp> | <Exp2>

		<Exp2> ::= <Exp3> `\textbar\textbar' <Exp2> | <Exp3>

		<Exp3> ::= <Exp4> `\&\&' <Exp3> | <Exp4>

		<Exp4> ::= <Exp5> (`==' | `\textless' | `\textgreater' | `\textless=' | `\textgreater=' | `!=') <Exp4> | <Exp5>

		<Exp5> ::= <Exp6> (`+' | `-') <Exp5> | <Exp6>

		<Exp6> ::= <ExpW> (`*' | `/' | `\%') <Exp6> | <ExpW>

		<ExpW> ::= <id> <Field>
			\alt <Op1> <Exp>
			\alt <int>
			\alt <char>
			\alt `False' | `True'
			\alt `(' <Exp> `)'
			\alt <FunCall>
			\alt `[]'
			\alt `(' <Exp> `,' <Exp> `)'
	\end{grammar}
\end{framed}

The \textsf{Exp} rule now starts with an attempt to read a cons operation.
If this fails, we fall through to the logical or operation, et cetera, until we ultimately reach what we call `weak' expressions, \textsf{ExpW}.

Clearly, these new rules have removed left recursion, while retaining expressiveness of expressions.
As an added bonus, the order of operations is now more clearly defined.

\subsection{Additional changes}

The \textsf{FunType} rule was modified to make its arrow optional for functions without any arguments:

\begin{framed}
	\begin{grammar}
	<FunType> ::= [ <FTypes> `\textrightarrow' ] <RetType>
	\end{grammar}
\end{framed}

\section{Parser Combinators}
\todo[inline]{parser combinators, how do they work?, megaparsec}

\section{Pretty Printing}
\todo[inline]{printer combinators}

\section{Reflection}
\todo[inline]{combinators are awesome. AST not fancy enough: records for extensibility, include locations for error messages}

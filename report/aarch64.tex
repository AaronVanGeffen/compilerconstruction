\chapter{AArch64}

AArch64 is the 64-bit instruction set supported by ARMv8 ARM CPUs.
In this section we will describe the architecture and how we tried to implement code generation.

\todo[inline]{Dibs Thom}

\section{Architecture Outline}

Our development platform is the ODROID-C2.
It is a Cortex-A53-based Android development board, but it can also run Linux.
It provides a quadcore CPU running at 1536 MHz and 2 gigabytes of RAM\@.

AArch64 provides 31 general-purpose registers $r0, \ldots, r30$.
This is excluding the stack pointer or the program counter.
Return addresses aren't put on the stack but done through the special \emph{link register} (r30).
This register can also be used as a callee-saved register where the value is written to memory.

In general, instructions on ARM have got distinct inputs and an output.
They take their inputs as registers and write their output to a register.
In AArch64 there are no special restrictions on the registers with regards to operations, i.e.\ each register can be used for every operation.
This allows for fewer restrictions in register allocation.

Registers can be addressed in 32-bit and 64-bit modes.
When accessed in a 32-bit fashion, the registers are written as $w0, \ldots, w30$.
When accessed in a 64-bit fashion, the registers are written as $x0, \ldots, x30$.
We will use the 64-bit registers in all generated code.

\section{Working on Heap}

The heap in programs managed by an operating system usually is subject to a lot more restrictions than we were under on the Simple Stack Machine.
It is not possible to write to arbitrary memory locations, because this will result in illegal memory access exceptions from the operating system.
Memory on the heap needs to be either be `baked in' to the binary or be requested through calls like \mintinline{c}{malloc}.

\subsection{Global Variables}

For global variables, the convention is to put them as labeled \texttt{.data} sections in the generated assembly code.
These can only be specified with static contents, however.
Because SPL allows for arbitrary expressions in the declaration of global variables, we move the initialisation code into a special function during code generation.
We rename the SPL \spl{main} function to \spl{_splmain} and insert a \spl{main} function that first calls our special \spl{_initGlobals} function.
It then calls the renamed \spl{_splmain} function to resume execution as usual.

We access global variables by having the program resolve the label to an address and loading or storing the value to or from that address.

\subsection{Heap-allocated objects}

Tuples and lists are allocated on the heap.
We create a special named section \texttt{.HEAP} in our assembly of a significant size.
Then we follow the same approach as with SSM\@: we put the address of this region in a register.
This register is then considered an ``end-of-heap''-pointer.
When a tuple or list is stored, we write the elements to the location pointed at by this pointer.
The end-of-heap-pointer register is copied to the variable register.
The pointer is then incremented, such that it points to free heap space again.

\subsection{Printing}

For printing \spl{Int} and \spl{Char} values we insert handwritten assembly functions into the generated code.
For integers, our function \spl{printInt(i)} calls libc's \mintinline{c}{printf} as \mintinline{c}{printf("%d\n", i)}

Characters are printed via a call to libc's \mintinline{c}{putwchar}.
This function supports unicode characters, so it allows us to output any character we could imagine.
Using this function does require to set the locale of the program.
We put this code in our special startup \mintinline{c}{main} function.

\section{Calling Convention}

We follow the C calling convention as defined in~\cite{callingconvention}.
This allows easier linking into \texttt{libc} methods.
We use these for printing numbers and variables.

The calling convention can be briefly summarised as follows.
The arguments to a function are passed via registers $r0, \ldots, r7$.
If there are more than seven arguments, the rest of the arguments are put on the stack.
The registers $r0, \ldots, r18$ are temporary or caller-saved registers.
A subroutine is free to clobber those registers.
If a calling function wishes to preserve their values, it needs to persist them to the stack.
The registers $r19, \ldots, r30$ are callee-saved.
This means that if a subroutine wishes to use them, it needs to save them to the stack.

\section{Register Allocation}

Register allocation is typically done by first transforming the IR of your program into a form that is known as \emph{Single Static Assignment}.
In this format, every variable is only assigned into once.
It is also vital to divide your program in \emph{Basic Blocks} with only one entry point and exit, so the allocation algorithm doesn't need to account for control flow such as jumps.
This information allows to generate better dependency graphs which then helps when applying the actual register allocation.
The register allocation is typically done via some graph colouring algorithm.
If there exists a $K$-colouring of the graph where $K$ is the number of available registers, it is possible to assign registers.
If this isn't possible, some values need to be \emph{spilled} to the stack and the algorithm is restarted.

Because transforming our IR to an SSA form did not appear to be trivial, we opted to ignore it and develop our own register allocation algorithm first.
This is a more quick-and-dirty version that would also simply not be able to handle too many registers.
Our algorithm can be briefly described as follows: first divide the program into basic blocks.
Then we determine what registers are in use for every line of IR\@.
This information is then used to assign registers.

We determine the lifecycle of variables in bottom-up fashion.
For each block, we start at the bottom with an empty list of live registers.
After all, at the end a function will have returned its result.
Then, for each line, we collect the inputs and the output pseudo-registers.
We remove the output from the live registers, and add the inputs to the live registers.
After all, before this line the output value apparently did not make sense, since it needed to be redefined.
Meanwhile the inputs must have existed before this line, otherwise they could not have been inputs.
We then annotate each line with the registers that are live at that point.

The obtained information about the lifetimes of the registers can then be easily used to determine a register allocation.
Since on every line it is known what registers are not needed anymore, we can drop reuse their assigned hardware registers.
That way we do not assign values to registers for longer than necessary.

\section{Incorporating the Calling Convention}

The register allocator needs to know what the inputs are.
In principle it should also be able to move them to other registers if necessary.
For every function entry we therefor generate a \asm{mov} for each argument, from the hardware register in which it is passed in to a pseudo-register named after the argument.

Before every function call we just spill all the caller-saved hardware registers and restore it after the function call.
After this, we again generate \asm{mov}s, this time from the pseudo-registers that hold the arguments to the hardware registers that they should go on.

Unfortunately, our register allocator currently is not capable of handling the extra constraints this puts on register allocation.
This leads to functions with multiple arguments sometimes getting their arguments mixed up.
This obviously leads to problems, ranging from wrong outputs to crashing programs.

\section{Remaining Work}

Aside from the register allocation the compiler to AArch64 is fairly complete.
For a proper register allocator we would probably need to revisit the separation into basic blocks.
They need better annotation about incoming and outgoing pseudo-registers.
Next we would need to implement the transformation to Static Single Assignment.
After that applying an algorithm to allocate registers based on graph colouring should be the next step.
The approach known as Iterated Register Coalescing by George and Appel~\cite{IRC} seemed promising.
It also includes a solution for the problem of determining what registers to spill, which is in itself an NP-hard problem.

After this futher a possible further improvement could be instruction scheduling, although a lack of documentation on the instruction characteristics of AArch64 makes this difficult.
Improving the toolchain to produce binaries instead of assembly code that still needs to be fed to a compiler like Clang or GCC would also improve the workflow.
We currently need them to figure out the linking to the used library subroutines.

\section{Reflection}

Generating the code was not that difficult, since our IR was already designed with ARM assembly in mind.
Register allocation proved a bit more challenging than expected.
Our initial attempts at getting our IR in an SSA form stranded when we had problems linking up basic blocks.
Abandoning proper register allocation worked great up until we tested functions with more arguments.

While the IR was convenient for transforming the more basic elements to assembly, we oversimplified some elements like \spl{if}-statements.
This lead to some inefficient code generation because we could not generate code with the test incorporated in the branching code.
Instead, we needed to do the compare, store its result and then branch based on a comparison with that result again.
A slightly higher-level IR would have allowed to do this in a better way, though perhaps at some cost of modularity.

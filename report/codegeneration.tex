\chapter{Code Generation}

This chapter details on what is arguably the ultimate goal of the compiler: the code generation process.
Using the type-checked abstract syntax tree, we generate an immediate representation.
In turn, that immediate representation, or IR, is then translated into assembly.

This chapter details the process from generating the IR to assembly code for the Simple Stack Machine, SSM.


\section{Intermediate Representation}

Since the end of the previous chapter, we have been able to verify whether a program is valid type-wise.
The program is still represented by the same AST we started out with during the parser phase, however.
To ease the assembly generation process, we first translated the AST to a data structure that looks more like actual assembly.

To start off, we require a few boilerplate data types that represent registers, labels, and immediates:

%\begin{listing}
\begin{minted}[frame=single]{Haskell}
data SplPseudoRegister
  -- simple register
  = Reg String
  -- ex: TupleLeft (TupleRight (Reg myTuple))
  | TupleFst SplPseudoRegister
  | TupleSnd SplPseudoRegister
  -- list value and pointer to tail
  | ListHd SplPseudoRegister
  | ListTl SplPseudoRegister
  -- ex: ListHd (TupleFst (Reg myTupleWithAListAndField))
  -- for tail empty
  | EmptyList
  deriving (Show, Eq)

type SplLabel = String

data SplImm = SplImmInt Integer | SplImmBool Bool | SplImmChar Char
  deriving Show
\end{minted}
%\caption{Definitions for pseudo-registers, labels and immediates.}
%\end{listing}

As registers are yet to be allocated in this phase, we introduce a pseudo-register data type.
We have opted to represent fields as encapsulated pseudo-registers, as they closely relate to load and store operations.
Furthermore, we introduce an \spl{EmptyList} register that will effectively yield a null-pointer.

Note that we will not be allocating many registers for the Simple Stack Machine (SSM), as most of the SSM's instructions operate directly on the stack.
However, as we will see in the next chapter, they are required to generate code for real hardware.

With these boilerplate types, we can start translating the program.
We will be using the following data type to represent it:

\begin{listing}
\begin{minted}[frame=single]{Haskell}
data SplInstruction
  = SplFunction SplLabel [SplPseudoRegister] SplIR
  -- if   unique   cond              Then  Else
  | SplIf SplLabel SplPseudoRegister SplIR SplIR
  -- while   unique   cond                       body
  | SplWhile SplLabel (SplIR, SplPseudoRegister) SplIR
  -- binop             op                dest
  | SplBinaryOperation SplBinaryOperator SplPseudoRegister
                    -- a                 b
                       SplPseudoRegister SplPseudoRegister
  -- unary op         op               dest
  | SplUnaryOperation SplUnaryOperator SplPseudoRegister
                   -- src
                      SplPseudoRegister
  | SplRet (Maybe SplPseudoRegister)
  -- mov   dest              src
  | SplMov SplPseudoRegister SplPseudoRegister
  | SplMovImm SplPseudoRegister SplImm
  | SplCall SplLabel (Maybe SplPseudoRegister)
            [SplPseudoRegister]
  -- make a new tuple     name           left
  | SplTupleConstr SplPseudoRegister SplPseudoRegister
                -- right
                   SplPseudoRegister

type SplIR = [SplInstruction]
\end{minted}
\caption{Definitions for instructions in IR.}
\end{listing}

With this, we can represent (sections of) the program as a list of \spl{SplInstruction}s, or \spl{SplIR} for short.
On first glance, the \spl{SplInstruction} data type still looks relatively high-level, of course.
However, looks can be deceiving.

The most notable important changes are found in the way expressions are transformed.
Expressive assignment statements are unrolled into multiple \spl{SplBinaryOperation}s and \spl{SplUnaryOperation}s as needed.
Note that all variable identifiers are replaced with pseudo-registers of the same name.
Intermediaries generated while unrolling are assigned to a pseudo-register with a unique name from a central pool.

In order to handle tuples efficiently tuples further down the line, we introduce an \spl{SplTupleConstr} construct.
This ensures the pair of (pseudo-)registers is kept together when allocating memory.

Finally, \spl{SplMov} and \spl{SplMovImm} are introduced to generate loads from registers and immediates, respectively.
These may, for example, be used to place arguments on the stack, before calling a function with an \spl{SplCall}.


\section{Simple Stack Machine}
\todo[inline]{Explain how registers aren't needed, how stack works, globals on stack as pseudo-heap, lists and tuples on heap, calling convention}


\section{Reflection}
\todo[inline]{IR too simple: loss of context, temporaries hard to distinguish (LDC, LDL) and thus aren't thrown away as they are treated like regular vars.}



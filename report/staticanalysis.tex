\chapter{Static Analysis}

In this chapter we will discuss how we approached the static analysis in our compiler.
We started out doing Type Checking, but later implemented the more advanced Type Inference.
We will then show the inference rules and reflect on our implementation.

\todo[inline]{Dibs Thom}
\todo[inline]{ergens scoping}

\section{Type Checking}

Type checking is a fairly simple analysis of a program.
By restricting the analysis to only verifying each specified type matches with what is expected, it is usually possible to treat each line by itself.
This does require a programmer to specify the types for each function or variable.
For some specific constructions, such as the empty list expression (\spl{[]}), this does require some special care: the type of the empty list is not as exactly defined as other constructions.

We implemented type checking by folding a function over the parsed syntax tree.
We track the state in a monad.
The functions recursively handle the nested segments of the declarations, functions, expressions and statements.
They either fail or return the result of the derivation of that segment.
Definitions of functions or variables are kept track of in the state with their derived type.
Encountered identifiers are looked up.

The data structures we used can be found in Listing~\ref{lst:typecheckdatastructures}.
We used several simple utility functions like \haskellinline{disallowVoid} to restrict the results of recursive \haskellinline{typeCheck} calls.
Listing~\ref{lst:typecheckunaryoperations} shows how this is applied for unary operations.

\begin{listing}[hbtp]
    \centering
    \inputhaskell{examples/typecheck_datastructures.hs}
    \caption{Data structures for type checking}\label{lst:typecheckdatastructures}
\end{listing}

\begin{listing}[hbtp]
    \centering
    \inputhaskell{examples/typecheck_unaryops.hs}
    \caption{Type checking unary operations}\label{lst:typecheckunaryoperations}
\end{listing}

\todo[inline]{How checked types}

\section{Type Inference}
\todo[inline]{Credit that tutorial, explain algorithm}

\section{Type Rules}

\section{Reflection}
\todo[inline]{Haskell is great. No error when specified type is more general than expression, time wasted on checking, poor error messages}


% vim: set ts=4 sw=4 tw=0 et wrap :
